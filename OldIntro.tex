
$137 \; fb^{-1}$, this number implies the number of collision events during the period of Run II at LHC. The physicists can expect the number of specific events by multiplying the cross-sections calculated by theorists by the $137 \; fb^{-1}$. HEP experimentalists do is put experimental data on the scale to find whether there are data from the signal processor or not. This process is known as observation. Isn't it artistic? Like large-scale modern arts, the integral collaboration of theorists, technicians, and experimentalists is connected fluidly, fraternally, and euphoniously.

Supersymmetry is a particle physics hypothesis that states that every particle possesses half-spin counterparts. This concept is important when we design models that go beyond the Standard Model. To be more exact, supersymmetry evolved as physicists built fundamental theories about our universe. However, any attempts to find models coupled with supersymmetry failed.

I had assumed that supersymmetry was a concept that ended in failure. However, what I learned in my master's coursework is precisely the opposite: it can and will be discovered, but we haven't found it yet. This is because the concept is not only the most essential topic on the pathway to the beyond Standard Model, but it is also constructed on a solid theoretical foundation that encourages experimentalists to seek the signal.

The most attractive point about supersymmetry is that it intrigues both theorists and experimentalists. For theorists, the notion of supersymmetry plays a footbridge to more profound ideas, such as supergravity and superstring, plus eventually to the AdS/CFT. In the experimental aspects, the hypothesis can explain outgrowth to many models which give solutions to unresolved problems that we have faced, like muon's g-2 and Higgs mass naturalness problems.

Its adaptability when encountering several problems of theory and experimental physics appeals to me. For example, the supersymmetric particle is not yet observed until mass near 2 TeV, and several models explain why we did not find such particles. This flexibility triggered me to want to build some unknown, invisible models and attain their signals on my own. 

%I pursue discovering evidence of new physics beyond the standard model(SM). Observation of various shreds of evidence in modern physics insists on new physics. However, until recently, no one has detected a direct signal from new physics candidates. I want to find incomparable evidence of new physics. Recent research implies that a new physics signal is underlying in the nature of leptons, and I hope to find the trail of supersymmetry dependent on lepton variables. The research experiences in my master's coursework benefit doing that search.

% and I hope to extend the idea to the search for supersymmetry. The first expected road stone to new physics is supersymmetry, so I wish to look for it.



%As an undergraduate student, I double-majored in mathematics and physics. Both majors positively influenced me when I scrutinized my disquisition. Excessive to say, physics, especially Quantum Field Theory, provided me with an inclusive narrative. Mathematics aided me in understanding the physical consequences of background events, as well as simulation and statistical data analysis. Linear algebra, for example, enables me to determine the corollaries of variables produced by machine learning approaches. In addition, the complex variable analysis assisted me in identifying methods for calculating amplitudes from scattering matrices via rotation on a complex plane. Furthermore, in my final semester of college, I enjoyed particle physics classes, which fascinated me because of their mathematical elegance.


%In period of master's coursework, I involved 2 researches collaborated with UCSB. R-parity violating supersymmetry search with professor David Stuart, and search for 4 top process with Valentina Duetta and Melissa Kathryn Quinnan. While I did the research, I noticed several points that UCSB's research facility suits for growing student's ability of research. 

%First, large number of faculty in UCSB would be helpful for research. Wide-ranged and keen professions scattered across the campus intelligently stimulates me, and it would be a great motivation to find research topics. Also, I could seek help more easily with this pool of people when I get stuck in an obstacle. For instance, when I had some problems with my computing environmental settings, I had to solve it my own and it consumed too much time to study. However, if there were experts of computing, then I would solve the problem quickly than before, and I would concentrate to research.

%In addition, research fits of professors in UCSB are comparable to my research goal. I hope to join the supersymmetry search, which can help us to understand the observed anomalies and theoretical problems, and lepton, especially neutrino, measurements, which are highly possible to possess the potent signature of new physics. In department of physics of UCSB, based on the experience I collaborated and the information gathered by website, there are several professors enrolled to experiments I have interested. 

%Finally, environmental condition of California helped me to concentrate to research. Mediterranean weather

