From undergraduate student to master in high energy physics, I have learned many things that would be advantageous when researching. During my undergraduate period, I double-majored in physics and mathematics. I am convinced that both majors will be helpful to researchers based on my experience. 

As a mathematics major, complex analysis and differential equations helped me when I studied the propagator of bosons and fermions. Numerical analysis lets me know how to fit some histograms to certain functions when dealing with the number of events. They were pretty helpful to me because they assisted me in understanding some statistical values on fitting results. 

As a physics major, quantum mechanics and classical mechanics taught me how some systems are defined and how they are quantized. Also, quantum field theory, which I studied three times separately until I got my master's degree, gave me helpful insight when researching. It allows me to understand SM and the background of most research, which helps me understand some vital research topics. For example, I can answer questions such as why a top quark always decays into a W boson and a bottom quark and why W and Z boson's kinematics are similar so we can exploit the Z's shape rather than the W boson by studying theoretical basis.
