
After finishing undergraduate school, I clutched a fantastic chance to join a master's new lab researching particle physics. The research experiences in that period give me invaluable learning for me.

At first, I participated in the measurement of a cross-section of 4 top quarks generated from the proton-proton collision in the CMS experiment by building strategies associated with trigger efficiency. The trigger depends on three variables, so we need to make a trigger efficiency scale factor with respect to three-dimensional space. This study taught me how the data analysis strategy was constructed and justified.

During my other master's research, I realized why supersymmetry is a groundbreaking notion in particle physics. I was taking part in a study looking for a signal from R-Parity violating supersymmetric particles. In this study, I helped to decide on a strategy, assess the method's rationality, and predict the outcomes of our approach. I studied the idea extensively while working on this project, which sparked my curiosity. The picture of supersymmetry dreaming impresses me because of its adaptability to explain physical anomalies and its naturalness. Also, I did most of this study and only left the signal unblinding process. I led trigger efficiency measurement, constructed the analysis strategy, measured uncertainties, and validated the strategy. This experience illuminated me how analysis works, and how to structure analysis. 
